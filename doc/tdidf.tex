Con la librería de \texttt{Gensim} se puede implementar la misma técnica de recuperación clasificada a través de la vectorización de documentos (del literal anterior) de forma sencilla. Para esto se parte del modelo de bolsa de palabras (BOW) construido en el procesamiento inicial. Posteriormente, se crea un modelo de Tf-idf (\texttt{TfidfModel()}) con el corpus de documentos y se transforma el modelo para evaluar la similaridad de \textit{queries}, creando un matriz de similaridad ((\texttt{MatrixSimilarity()}) con el modelo de Tf-idf del corpus de documentos. Este proceso lo puede encontrar en el cuaderno \texttt{HW01\_6.ipybn}. \\

Ahora bien, con esta matriz de similaridad y el modelo de Tfidf se transforman cada una de las \textit{queries} y se realiza la recuperación de la información a partir de su similaridad de coseno (\textit{cosine similarity score)}, ordenando los resultados de mayor a menor similaridad. De igual forma, se tienen en cuenta únicamente los documentos con un puntaje (\textit{score}) mayor a 0. Estos se exportan al archivo \texttt{GENSIM-queries\_results}, con el mismo formato del archivo de etiquetas para su posterior evaluación (véase siguiente sección).