Las dos técnicas de IR presentadas anteriormente consideran únicamente una selección binaria de documentos (estos son o no son relevantes para el \textit{query}). Esta no suele ser una buena estrategia por lo que es difícil para los usuarios construir \textit{queries} booleanos, se sufre del problema de \textit{feast or famine} (muchos o muy pocos resultados) y cuando son muchos todos los documentos tienen un mismo nivel de relevancia. Con esto en mente, con la técnica de recuperación clasificada (\textit{ranked retrieval}), lo que se desea es no solo encontrar los documentos relevantes para el \textit{query} sino también poder ordenarlos de alguna forma. \\

Para esta implementación básica de \textit{ranked retrieval} se consideran dos factores: la frecuencia de un termino en el documento (cuantificado por el \textit{term frequency} o $tf$) y la rareza de la palabra entre los documentos (cuantificado por el \textit{inverse document frecuency} o $idf$). Por un lado, se asume que, entre más veces estén los términos de un \textit{query} en un documento, este es más relevante. Y, por el otro lado, se asume que, la rareza de los términos en la colección los hace más importantes (son más informativos). De esta manera, se utilizan estos dos conceptos para dar un puntaje (\textit{score}) a cada uno de los documentos de la colección sobre los términos de un \textit{query}. \\

Ahora bien, la importancia de la cantidad de términos en un documento (\textit{term frecuency}) no necesariamente es lineal (si un término del \textit{query} aparece 10 veces en un documento, este no es necesariamente 10 veces más importante que un documento que solo lo tiene 1 vez). Por esta razón, típicamente se utiliza la frecuencia logarítmica para pesar este concepto:

\begin{equation}
    w_{t,d} = \begin{cases} 
            1 + log_{10}( tf_{t,d} ) &\mbox{if } tf_{t,d} > 0  \\
            0 & \mbox{otherwise }  
            \end{cases}
    \label{tf}
\end{equation}

Por su parte, la cantidad de documentos en los que aparece un termino (\textit{document frecuency} o $df$) es una medida inversa de la rareza, por lo que se utiliza su forma invertida \textit{inverse document frecuency} o $idf$). A esta medida, de forma similar a la anterior, se le aplica la función logaritmo para reducir el efecto que tiene:

\begin{equation}
    idf_t = log_{10} \left(\frac{N}{df_t}\right)
    \label{idf}
\end{equation}

Así las cosas, para obtener el peso de cada término para cada documento simplemente se multiplican los dos pesos explicados en (\ref{tf}) y (\ref{idf}). COn esto se obtiene: 

\begin{equation}
    w_{t,d} = tf * idf_{t,d} = log(1 + tf_{t,d}) * log_{10}\left(\frac{N}{df_t}\right)
    \label{eq:tfidf}
\end{equation}

Donde:
\begin{itemize}
    \item $w$ es el peso resultante.
    \item $t$ es el índice del término.
    \item $d$ es el índice del documento.
    \item $tf_{t,d}$ es la frecuencia del término $t$ en el documento $d$.
    \item $df_{t}$ es la frecuencia de documento. Corresponde al número de documentos que contienen el término $t$. 
    \item $idf$ es la frecuencia de documento invertida. Corresponde al total inverso de documentos que contienen el término $t$ al menos una vez.
    \item $N$: Tamaño total del corpus.
\end{itemize}

Finalmente, para obtener el puntaje (o \textit{score}) de cada documento para un \textit{query} dado, lo único que se debe hacer es sumar estos pesos sobre todos los términos de dicho \textit{query}.

\begin{equation}
    score(q,d) = \sum_{t\in q \cap d} tf * idf_{t,d}
\end{equation}

\subsubsection{Implementación}

La implementación de esta estrategia de recuperación de información se encuentra en el cuaderno \texttt{HW01\_4.ipybn}. Para esta se decidió construir la matriz de $tf$ a partir del corpus de documentos, en donde se utiliza la expresión (\ref{tf}). Y, de igual forma, se construyó el vector de $idf$, con la expresión (\ref{idf}) para cada termino del vocabulario. A partir de estas dos expresiones se crea la matriz de pesos \textit{tf-idf} con la que se procede a realizar los \textit{queries} del dataset. \\

Así las cosas, se construye la función \textit{basic\_ranked\_retrieval(tfidf_matrix, query)}, la cual recibe como parámetros la matriz con los pesos del corpus \textit{tfidf} y el query correspondiente, y retorna los documentos documentos relevantes de cada query. Para esto se tienen en cuenta únicamente los documentos con un puntaje (\textit{score(q,d)}) mayor a 0. Estos se exportan al archivo \texttt{RRI-queries\_results}, con el mismo formato del archivo de etiquetas para su posterior evaluación.