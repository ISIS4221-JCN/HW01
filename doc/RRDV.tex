La técnica de recuperación clasificada con vectorización de documentos se basa en asignar una calificación a la similaridad a dos vectores, uno de los cuales corresponde a un documento y el otro a la consulta. La similaridad entre los dos vectores se obtiene con base en la similaridad el coseno, pues otro tipo de distancias como Euclidiana o Manhattan no permitirían obtener un valor adecuando. La similaridad del coseno está dada por la ecuación \ref{eq:cosine}.
\begin{equation}
    cos(\vec{q}, \vec{d}) = \frac{\sum_{i=1}^{|V|}q_id_i}{\sqrt{\sum_{i=1}^{|V|}q_i^2} \sqrt{\sum_{i=1}^{|V|}d_i^2}}
    \label{eq:cosine}
\end{equation}

Allí $q$ hace referencia a la consulta, por su inicial en ingles (\textit{query}) y $d$ hace referencia al documento con el que se está comparando la consulta. $|V|$ hace referencia al tamaño del vocabulario, el cual, en este caso es de m.\\

El proceso de vectorización de cada documento se obtiene con base en el modelo \textit{tf-idf}. Este indica que cada documento o consulta se representa con un vector con tamaño del vocabulario en donde cada registro indica un peso asignado a un término específico en relación a ese documento. Cada peso se obtiene con base en la ecuación \ref{eq:tfidf}, explicada en detalle en la sección anterior. \\

Como retorno se organiza los documentos según la similaridad que presentan. Podría tenerse en cuenta un umbral para descartar documentos con baja similaridad.

\subsubsection{Implementación}
Como se solicita en las instrucciones, se inicia con la implementación de una función que permita obtener el vector que representa cada documento con base en el peso $w_{t,d}$. Para ello, se inicia creando una matriz en donde se registra qué términos contiene cada documento con el fin de obtener un vector $idf$, el cual se multiplica con el logaritmo de la suma de 1 con la frecuencia del término asociado en el documento correspondiente.\\

Esta función se utiliza para calcular la similaridad del coseno entre el vector retornado para la consulta y para el documento. Finalmente, se organizan los valores retornados para cada documento para seleccionar aquellos que resultan relevantes.