La búsqueda binaria es un algoritmo en que se utilizan operaciones lógicas como AND y OR con el fin de encontrar documentos que contienen los términos dados en la consulta, asumiendo que ello indica que son relevantes para dicha necesidad de información. Para llevar a cabo la búsqueda de manera eficiente, se crea una matriz binaria indicando los términos que contiene cada documento, dando el nombre de búsqueda binaria. A continuación, se explica más detalladamente la técnica.

\subsubsection{Matriz binaria}
Una vez se conoce de cada documento los términos que contiene y su frecuencia, se procede a construir la matriz binaria. Las filas de esta matriz corresponden a los documentos que se consideran en la búsqueda y las filas corresponden a los términos existentes en el diccionario. Siendo así, los registros de cada columna que corresponden a uno indican que ese término existe en el documento indicado por la columna. \\